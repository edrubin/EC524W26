\documentclass[10pt]{article}
\usepackage{lmodern}
\usepackage{amssymb,amsmath}
% \usepackage{fontspec}

\usepackage[margin=1.15in]{geometry}
\usepackage{setspace, titling}
\newcommand{\subtitle}[1]{%
  \posttitle{%
    \par\end{center}
    \begin{center}\large#1\end{center}
    \vskip0.5em}%
}

%% FONTS
\usepackage{fontspec}
\setmainfont{Fira Sans Extra Condensed} %
\setmonofont{Fira Code}
\usepackage{listings, lstfiracode}
\usepackage{marvosym} % For cool symbols.
\usepackage{fontawesome} % Ditto

\usepackage[dvipsnames]{xcolor}
\definecolor{uo_green}{HTML}{154733}
\definecolor{forest_green}{HTML}{006241}
\definecolor{pine_green}{HTML}{007935}
\definecolor{grass_green}{HTML}{62A70F}
\definecolor{golden_yellow}{HTML}{FFD200}
\definecolor{cool_gray}{HTML}{54565B}
\definecolor{light_cool_gray}{HTML}{A8A8AA}

\usepackage[colorlinks = true,
linkcolor = pine_green,
urlcolor  = pine_green,
citecolor = pine_green,
anchorcolor = black]{hyperref}
\usepackage{graphicx}

% For strikeout font: \sout()
\usepackage[normalem]{ulem}
% New emphasis style: Bold and underlined
\newcommand{\emf}[1]{\textbf{\textcolor{grass_green}{#1}}}

% For table formatting:
\usepackage{array, booktabs, caption, siunitx}
\newcommand{\ra}[1]{\renewcommand{\arraystretch}{#1}}
\newcolumntype{d}[1]{D{.}{.}{#1}}

\begin{document}

\title{
  % \vspace{-4em}
	\texttt{\textbf{Prediction and machine learning} [EC524/424]} \\[1em]
	\large Winter 2026 Syllabus \\
  \href{https://github.com/edrubin/EC524W26/}{\normalsize \texttt{https://github.com/edrubin/EC524W26}}
}
\author{\textbf{Dr. Edward Rubin}\\ Dept. of Economics, University of Oregon}
\date{}

\maketitle

\vspace*{-4ex}

% \section*{Basics}

\begin{table}[!ht]
	% \centering
	\ra{1.2}
\begin{tabular}{@{\extracolsep{5pt}} lll @{}}
	& \underline{\textbf{{Instructor}}} & \underline{\textbf{{GE}}}\\
	\faUser & \emf{Edward Rubin} & \emf{Jose Rojas Fallas}\\
	\faPaperPlaneO & \href{mailto:edwardr@uoregon.edu?subject=EC524}{edwardr@uoregon.edu} & \href{mailto:jrojas2@uoregon.edu?subject=EC524}{jrojas2@uoregon.edu}\\
	 & \multicolumn{2}{c}{Start subject with ``\texttt{EC524:}''.} \\
  \faBuildingO & PLC 530 & PLC 525 \\
  \faHourglassStart & Tu. 2:00p--3:30p & TBA \\
  \faChevronRight & \href{https://edrub.in}{edrub.in} & \href{https://jose-rojas-fallas.quarto.pub/}{jose-rojas-fallas.quarto.pub} \\
\end{tabular}
\end{table}

\noindent \textbf{Email note:} We will do our best to respond promptly to your emails. Our responses may be slower over weekends/holidays. There may be times that our responses take up to 48 hours. Please do not repeatedly send the same email unless it has been more than 48 hours.

\begin{table}[!ht]
	\ra{1.2}
\begin{tabular}{@{\extracolsep{5pt}} l l l l l l @{}}
	& \underline{\textbf{{Lecture}}} & \underline{\textbf{{Lab}}} \\
	\faClockO & Tu. \& Th., 10:00a--11:20a & Fr., 10:00a--10:50a & \\
  \faMapMarker & \href{https://classrooms.uoregon.edu/esslinger-105/}{105 Esslinger} & \href{https://classrooms.uoregon.edu/plc-72/}{072 PLC} \\
	\faUser & Ed & Jose $|$ Ed  \\
  \faChevronRight & \multicolumn{2}{l}{\emf{Our class:} \href{https://github.com/edrubin/EC524W26/}{https://github.com/edrubin/EC524W26/}} \\
  \faChevronRight & \multicolumn{2}{l}{2025: \href{https://github.com/edrubin/EC524W25/}{https://github.com/edrubin/EC524W25/}} \\
  \faChevronRight & \multicolumn{2}{l}{2024: \href{https://github.com/edrubin/EC524S24/}{https://github.com/edrubin/EC524S24/}} \\
  \faChevronRight & \multicolumn{2}{l}{2023: \href{https://github.com/edrubin/EC524W23/}{https://github.com/edrubin/EC524W23/}} \\
  \faChevronRight & \multicolumn{2}{l}{2022: \href{https://github.com/edrubin/EC524W22/}{https://github.com/edrubin/EC524W22/}} \\
  \faChevronRight & \multicolumn{2}{l}{2021: \href{https://github.com/edrubin/EC524W21/}{https://github.com/edrubin/EC524W21/}} \\
  \faChevronRight & \multicolumn{2}{l}{2020: \href{https://github.com/edrubin/EC524W20/}{https://github.com/edrubin/EC524W20/}}
\end{tabular}
\end{table}

\newpage

\section*{Course summary}

\paragraph{Description} Following the first course on econometrics and causal inference in our sequence, EC524 turns to examining the \emf{tools available and best practices for predicting outcomes}. Put simply, we are now focusing on $\hat{y}_i$ rather than $\hat{\beta}$ from the model $y_i = \alpha + \beta x_i + \varepsilon_i$.

Learning statistical programming is inherent to practicing applied econometrics. Consequently, throughout this course we will also teach the statistical programming language \texttt{R}.

\paragraph{Objectives}

\begin{enumerate}
  \item \emf{Distinguish} between settings that require \emf{causal inference} \textit{vs.} settings that want \emf{prediction}.
  \item Understand the main \emf{themes and best practices} in modern \emf{prediction} methods/contexts.
  \item Develop \emf{familiarity} with common machine-learning algorithms---and their strengths/weaknesses.
  \item Build \emf{intuition} for prediction---especially the bias-variance tradeoff.
  \item Expand \emf{R expertise}.
\end{enumerate}

\paragraph{Prerequisites} This course requires the previous course in our sequence---\textit{i.e.}, Economics 423/523. I also assume you are comfortable in \texttt{R}.

\section*{Books} I know you are busy and reading for class is often difficult.\newline However, \emf{if you are actually here to learn, then read these books}.

\bigskip

\noindent\textit{Note} Each book (except two of the recommended books) is available for \emf{free online}. The physical copies are also very reasonably priced---I suggest you buy physical versions for books that you like.

\paragraph{Required books}

\begin{enumerate}
\item \href{https://www.statlearning.com/}{Introduction to Statistical Learning} \textit{ISL}
\item \href{http://themlbook.com/}{The Hundred-Page Machine Learning Book} \textit{100ML}
\item \href{https://socviz.co/}{Data Visualization} \textit{Data Viz}
\end{enumerate}

\paragraph{Suggested books}

\begin{enumerate}
\item \href{https://r4ds.had.co.nz/}{R for Data Science} \textit{RDS}
\item \href{https://www.springer.com/us/book/9783319500164}{Introduction to Data Science} \textit{IDS} (not available without purchase)
\item \href{http://web.stanford.edu/~hastie/ElemStatLearn/}{The Elements of Statistical Learning} \textit{ESL} (the big brother of \textit{ISL})
\item \href{https://link.springer.com/book/10.1007/978-3-030-71352-2}{Data Science for Public Policy} \textit{DSPP} (e-book available via UO library)

\end{enumerate}

\section*{Software and tools}

\begin{itemize}
  \item We will use the statistical programming language \href{https://www.r-project.org/}{\textbf{\texttt{R}}}.
  \item We will use \href{https://www.rstudio.com}{\textbf{\texttt{RStudio}}} to interact with \texttt{R}.
\end{itemize}
Learning \texttt{R} will require time and effort, but it is a powerful and versatile tool that is valued by many employers. Put in the requisite effort and time, and you will be rewarded.

\section*{Labs, assignments, projects, and exams}

\paragraph{Attend the lab} This course includes a lab, which is \emf{integral to learning} the material in (and passing) this course. The lab includes both general econometrics instruction and computing resources necessary to complete the course and learn/master its topics.

\paragraph{Assignments}
\begin{itemize}
  \item You will submit \emf{typed assignments via Canvas}, generally in one of two formats (we'll tell you what we want):
  \begin{enumerate}
    \item An \href{https://www.rstudio.com/blog/r-notebooks/}{R notebook} that is hosted somewhere on the web or submitted as a self-contained HTML file;
    \item A link to a \href{https://www.kaggle.com/docs/notebooks}{Kaggle notebook}.
  \end{enumerate}
  % \item Assignments will typically be due on Thursday evenings.
  \item We will grade on a \emf{complete/incomplete scale}. \newline Low-quality work will be returned to be re-submitted as late.
\end{itemize}

\paragraph{Late submissions} Students whose assignments are occasionally late will be penalized half a letter grade. Students whose assignments are frequently late will be penalized a full letter grade.

\paragraph{Group work} Feel free to work together on the assignments. Unless explicitly stated, each student is required to write and submit independent answer sheets. This means that word-for-word copies will not be accepted and will be viewed as academic dishonesty. If you work with other students, you must list the students in your study group at the top of your assignment. If you fail to do so, you will receive a score of zero.

\paragraph{Project 1: Application} (Due 04 March 2026) For this project, you will apply what you've learned in this course to a prediction problem/context of your choice. You will:
\begin{itemize}
  \item Find data for your problem,
  \item Clean data as needed,
  \item Use multiple prediction models and best practices to make predictions,
  \item Write up everything in a clean notebook/blog,
  \item Present a five-minute presentation on the process/results.
\end{itemize}

\paragraph{Project 2: Extension} (Due 11 March 2026) This project pushes you to extend your knowledge to a new method. You will choose a topic that we have not covered in class---but that is related to topics covered in class---\textit{e.g.}, spectral clustering or time-series prediction. You will:
\begin{itemize}
  \item Learn about this topic on your own,
  \item Write a `wiki` that explains this topic using \textit{math and examples},
  \item Make and give a five-minute presentation on the topic.
\end{itemize}
No duplicates for topics. I will provide a list of ideas on the course site. The idea here is that you extend/apply the course's ideas to situations that \textbf{interest you}.

\paragraph{Exams} We will proctor an \emf{in-person \href{https://registrar.uoregon.edu/calendars/examinations}{final}} on Monday, March 17\textsuperscript{th}, 2026 from 8:00a--10:00a Pacific. We \textbf{will not} offer remote or make-up options for the exam.

\section*{Recommendations}

\begin{enumerate}
  \item \emf{Be kind}.
  \item \emf{Take responsibility} for your own education and try to \emf{learn} as much as you can.
  \item \emf{Do your own work}.
  \item Develop your \emf{intuition}---\textit{e.g.}, why would method $x$ work in one situation and fail in another?
  \item \emf{Learn \texttt{R}}. Struggle while you try---and use \emf{Google} to figure things out.
  \item Come to \emf{office hours}.\footnote{Two related articles from NPR on office hours: \href{https://www.npr.org/2019/10/05/678815966/college-students-how-to-make-office-hours-less-scary}{\textit{College Students: How to Make Office Hours Less Scary}} and \href{https://www.npr.org/2019/10/02/766568824/uncovering-a-huge-mystery-of-college-office-hours}{\textit{Uncovering A Huge Mystery Of College: Office Hours}}.}
\end{enumerate}

\section*{Honesty and academic integrity}

\emf{You must do your own work.} Do not claim credit for any work other than your own. \emf{Your work should not be identical to others' work.} Cheating or plagiarizing of any sort on any component of this class will result in a failing grade for the term and a report of the offense to the university. Please acquaint yourself with the \href{http://studentlife.uoregon.edu}{Student Conduct Code}.

\textbf{Large language models} (and other sources): ChatGPT, GitHub Copilot, Claude, and other related AI `assistants' are great tools. I am totally fine with you using them---and even encourage it. However, you still need to submit work \emf{in your own words}, and you need to \emf{understand the code} that you submit. Anything less is plagiarism, lazy, and a loss of opportunity to actually learn valuable material/tools.

\section*{Accessibility}

The University of Oregon and I are dedicated to fostering inclusive, equitable, and accessible learning environments for all students. The Accessible Education Center (AEC) assists students with disabilities in reducing barriers in the educational experience. You may be eligible for accommodations for a variety of disabilities – apparent disabilities, such as a mobility or physical disability, or non apparent disabilities, such as chronic illnesses or psychological disabilities. \href{https://aec.uoregon.edu/content/what-disability}{If you have or think you have a disability} and experience academic barriers, please contact the Accessible Education Center (Location: 360 Oregon Hall; 541-346-1155; \href{https://aec.uoregon.edu/}{uoaec@uoregon.edu}) to discuss appropriate accommodations or support. The details of your disability will be kept confidential with the AEC and you are not expected to share this information with others. However, I invite you to discuss any approved accommodations or access needs at any time with me.

\section*{Grading}

Grades will be assigned as follows.\footnote{Undergraduates are allowed to miss one additional assignment in the scheme.}

\begin{table}[!ht]
  \ra{1.5}
  \begin{tabular}{@{\extracolsep{1cm}} cccc @{}}
    \textbf{\underline{Grade}} & \textbf{\underline{Assignments}} & \textbf{\underline{min(Projects)}} & \textbf{\underline{Final exam}} \\
    \textbf{A}
    & \textit{Incomplete} on 0 assignments.
    & $\geq$ \textit{Professional}
    & $\geq$ 80\% \\
    \textbf{B}
    & \textit{Incomplete} on $\leq$ 1 assignments.
    & $\geq$ \textit{Minor revision}
    & $\geq$ 70\% \\
    \textbf{C}
    & \textit{Incomplete} on $\leq$ 2 assignments.
    & $\geq$ \textit{Moderate revision}
    & $\geq$ 60\% \\
    \textbf{D}
    & \textit{Incomplete} on $\leq$ 3 assignments.
    & $\geq$ \textit{Major revision}
    & $\geq$ 50\% \\
  \end{tabular}
\end{table}
\noindent Recall that assignments are graded as \textit{Complete} \textit{vs.} \textit{Incomplete}---the standard for \textit{Complete} is much higher than simply submitting.

\newpage

\section*{Tentative, overly-ambitious, predicted outline}

\textit{Note:} Stay up to date on our class \href{https://github.com/edrubin/EC524W26/}{class's Github page}.

\paragraph{0. An introduction to prediction and statistical learning}
\begin{enumerate}
  \item What are we doing? \textbf{Readings} \textit{ISL} Introduction, Ch1
  \item Prediction \textit{vs.} causal inference \textbf{Readings} \href{https://www.aeaweb.org/articles?id=10.1257/aer.p20151023}{\textit{Prediction Policy Problems}} by Kleinberg \textit{et al.} (2015)
  \item Modeling decisions and assessment \textbf{Readings} \textit{ISL} Ch3
\end{enumerate}

\paragraph{1. Exploratory data analysis}
\begin{enumerate}
  \item Building insights from graphics \textbf{Readings} \textit{Data Viz} Preface, Ch1
  \item \texttt{ggplot2} \textbf{Readings} \textit{Data Viz} Ch3
\end{enumerate}

\paragraph{2. Supervised learning}
\begin{enumerate}
  \item An introduction to machine learning \textbf{Readings} \textit{100ML} Preface, Ch1--Ch4; \textit{ISL} 2.1--2.2
  \item Resampling methods and other best practices \textbf{Readings} \textit{100ML} Ch5; \textit{ISL} Ch5
  \item Why don't we stick with regression? \textbf{Readings} \textit{ISL} Ch3
  \item LASSO and Ridge regression \textbf{Readings} \textit{ISL} 6.1--6.3, 6.6
  \item Classification and logistic regression \textbf{Readings} \textit{ISL} 4.1--4.3
  \item Decision trees \textbf{Readings} \textit{100ML} 3.3; \textit{ISL} 8.1
  \item Ensembles: Bagging, random forests, boosting \textbf{Readings} \textit{ISL} 8.2--8.3 \textit{100ML} 7.5 and Ch8
  \item SVM \textbf{Readings} \textit{100ML} 3.4; \textit{ISL} 9.1--9.4
  \item Neural nets \textbf{Readings} \textit{100ML} 6
  \item Additional topics \textbf{Readings} \textit{100ML} Ch7 anc Ch11
\end{enumerate}

\paragraph{3. Unsupervised learning}
\begin{enumerate}
  \item Introduction to unsupervised learning \textbf{Readings} \textit{100ML} Ch9; \textit{ISL} 10.1
  \item Principal components analysis \textbf{Readings} \textit{ISL} 10.2; \textit{100ML} 9.3
  \item Nearest-neighbor matching, \textit{K}-means, and hierarchical clustering \textbf{Readings} \textit{100ML} Ch9; \textit{ISL} 10.3
\end{enumerate}

\paragraph{4. Extensions}
\begin{enumerate}
  \item Bias and fairness \textbf{Readings} \href{https://www.technologyreview.com/s/612876/this-is-how-ai-bias-really-happensand-why-its-so-hard-to-fix/}{Hao (2019)}
\end{enumerate}

\end{document}
